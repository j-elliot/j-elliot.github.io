%% start of file `template.tex'.
%% Copyright 2006-2013 Xavier Danaux (xdanaux@gmail.com).
%
% This work may be distributed and/or modified under the
% conditions of the LaTeX Project Public License version 1.3c,
% available at http://www.latex-project.org/lppl/.


\documentclass[11pt,a4paper,sans]{moderncv}        % possible options include font size ('10pt', '11pt' and '12pt'), paper size ('a4paper', 'letterpaper', 'a5paper', 'legalpaper', 'executivepaper' and 'landscape') and font family ('sans' and 'roman')

% moderncv themes
\moderncvstyle{oldstyle}                           % style options are 'casual' (default), 'classic', 'oldstyle' and 'banking'
\moderncvcolor{grey}                               % color options 'blue' (default), 'orange', 'green', 'red', 'purple', 'grey' and 'black'
%\renewcommand{\familydefault}{\sfdefault}         % to set the default font; use '\sfdefault' for the default sans serif font, '\rmdefault' for the default roman one, or any tex font name
%\nopagenumbers{}                                  % uncomment to suppress automatic page numbering for CVs longer than one page

% character encoding
\usepackage[utf8]{inputenc}                       % if you are not using xelatex ou lualatex, replace by the encoding you are using
%\usepackage{CJKutf8}                              % if you need to use CJK to typeset your resume in Chinese, Japanese or Korean

% adjust the page margins
\usepackage[scale=0.75]{geometry}
%\setlength{\hintscolumnwidth}{3cm}                % if you want to change the width of the column with the dates
%\setlength{\makecvtitlenamewidth}{10cm}           % for the 'classic' style, if you want to force the width allocated to your name and avoid line breaks. be careful though, the length is normally calculated to avoid any overlap with your personal info; use this at your own typographical risks...

% personal data
\name{J. Elliot}{Miller}
\title{Resumé title}                               % optional, remove / comment the line if not wanted
\address{18 Nelson Hill Road}{Wassaic, NY}{}% optional, remove / comment the line if not wanted; the "postcode city" and and "country" arguments can be omitted or provided empty
\phone[mobile]{+1~(717)~398~1540}                   % optional, remove / comment the line if not wanted
%\phone[fixed]{+2~(345)~678~901}                    % optional, remove / comment the line if not wanted
%\phone[fax]{+3~(456)~789~012}                      % optional, remove / comment the line if not wanted
\email{jem060@bucknell.edu}                               % optional, remove / comment the line if not wanted
\homepage{{j-elliot.github.io}}                         % optional, remove / comment the line if not wanted
\extrainfo{}                 % optional, remove / comment the line if not wanted
\photo[64pt][0.4pt]{picture}                       % optional, remove / comment the line if not wanted; '64pt' is the height the picture must be resized to, 0.4pt is the thickness of the frame around it (put it to 0pt for no frame) and 'picture' is the name of the picture file
\quote{Some quote}                                 % optional, remove / comment the line if not wanted

% to show numerical labels in the bibliography (default is to show no labels); only useful if you make citations in your resume
%\makeatletter
%\renewcommand*{\bibliographyitemlabel}{\@biblabel{\arabic{enumiv}}}
%\makeatother
%\renewcommand*{\bibliographyitemlabel}{[\arabic{enumiv}]}% CONSIDER REPLACING THE ABOVE BY THIS

% bibliography with mutiple entries
%\usepackage{multibib}
%\newcites{book,misc}{{Books},{Others}}
%----------------------------------------------------------------------------------
%            content
%----------------------------------------------------------------------------------
\begin{document}
%-----       letter       ---------------------------------------------------------
% recipient data
\recipient{Pasona N A, Inc. Recruitment team}{Pasona N A, Inc.\\New York, NY}
\date{19 December, 2023}
\opening{Dear Hiring Manager,}
\closing{Yours faithfully,}         % use an optional argument to use a string other than "Enclosure", or redefine \enclname
\makelettertitle

I am writing to express my keen interest in the Associate Software Engineer position at Pasona. With a solid background in programming, technical support, and project collaboration, I am excited about the opportunity to contribute my skills and expertise to your dynamic team.

As an entrepreneur responsible for the business software environments for my companies, I honed my programming skills, particularly in C\#, Python, and VB, and gained hands-on experience with SQL, and GitHub. My ability to manage the entire software development life cycle aligns seamlessly with the accountabilities outlined for the Associate Software Engineer role.

Throughout my career, I have successfully executed tasks according to defined schedules, provided technical support to end-users, and collaborated with project managers to ensure the smooth progression of projects. My strong problem-solving and analytical skills have enabled me to interpret business requirements into IT specifications effectively.

The prospect of contributing to the configuration, development, testing, deployment, and ongoing maintenance of software applications excites me. I am confident that my commitment to staying updated with industry IT trends, conducting research, and supporting proof-of-concept initiatives aligns with Pasona's focus on delivering innovative solutions to customers.

I am particularly drawn to Pasona's commitment to excellence, as reflected in the emphasis on planning and organizational skills, team-oriented approach, and continuous learning. The hybrid work model, coupled with the opportunity to contribute to the Business Service Division, aligns perfectly with my career goals.

Thank you for considering my application. I am enthusiastic about the opportunity to contribute to Pasona's success and look forward to the possibility of discussing how my skills align with the requirements of the Associate Software Engineer position.

\makeletterclosing

\end{document}


%% end of file `template.tex'.
